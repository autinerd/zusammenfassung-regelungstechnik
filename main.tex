% !TeX encoding = UTF-8
% chktex-file 21 chktex-file -2 chktex-file 23 chktex-file 1 chktex-file 6
\documentclass[11pt,a4paper,fleqn,headsepline]{scrreprt}
\usepackage[utf8]{inputenc}
\usepackage[ngerman]{babel}
\usepackage{amsmath}
\usepackage{amsfonts}
\usepackage{amssymb}
\usepackage{enumitem}
\usepackage{gensymb}
\usepackage{multicol}
\usepackage{geometry}
\usepackage[autostyle=true,german=quotes]{csquotes}
\usepackage[pdftex,pdfa,hidelinks]{hyperref}
\geometry{left=2cm,top=1.5cm,bottom=1cm,right=1cm}

\begin{document}
    \begin{multicols}{2}

\renewcommand{\cleardoublepage}{}
\renewcommand{\clearpage}{}

        \section*{Routh-Hurwitz}
        Form: \(a_0 s^n + a_1 s^{n-1} + \cdots + a_{n-2} s^2 + a_{n-1} s + a_n\)

        Hurwitz-Matrix (Spalten- und Zeilenanzahl entspricht höchster Potenz):
        \[\begin{bmatrix}
            a_1 & a_3 & a_5 & \cdots \\
            a_0 & a_2 & a_4 & \cdots \\
            0 & a_1 & a_3 & \cdots \\
            \vdots & \vdots & \vdots & \ddots \\
        \end{bmatrix}\]

        Jeder Schritt nach rechts: \(a_{n+2}\), jeder Schritt nach unten: \(a_{n-1}\)

        System stabil wenn alle nordwestlichen Unterdeterminanten \(>0\):
        \begin{align*}
            H_1 &= \begin{vmatrix} a_1 \end{vmatrix} > 0 \\
            H_2 &= \begin{vmatrix} a_1 & a_3 \\ a_0 & a_2 \end{vmatrix} > 0 \\
            H_3 &= \begin{vmatrix} a_1 & a_3 & a_5 \\ a_0 & a_2 & a_4 \\ 0 & a_1 & a_3 \end{vmatrix} > 0 \\
            \text{etc.}
        \end{align*}

        Beispiel: \(s^4+s^3+2s^2+2s+1\)

        \(a_0 = 1; a_1 = 1; a_2 = 2; a_3 = 2; a_4 = 1\)

        Hurwitz-Matrix:
        \[\begin{bmatrix}
            1 & 2 & 0 & 0 \\
            1 & 2 & 1 & 0 \\
            0 & 1 & 2 & 0 \\
            0 & 1 & 2 & 1 \\
        \end{bmatrix}\]

        System stabil wenn alle nordwestlichen Unterdeterminanten \(>0\):
        \begin{align*}
            H_1 &= \begin{vmatrix} 1 \end{vmatrix} > 0 \\
            H_2 &= \begin{vmatrix} 1 & 2 \\ 1 & 2 \end{vmatrix} = 0 \ngtr 0 \\
            &\text{nicht stabil}
        \end{align*}

        \section*{Zustandsraumdarstellung}
        \begin{align*}
            \mathbf{A} \in \mathbb{R}^{n \times n} &= \begin{bmatrix}
                a_{11} & a_{12} & \cdots \\
                a_{21} & a_{22} & \cdots \\
                \vdots & \vdots & \ddots \\
            \end{bmatrix} \\
            \mathbf{B} \in \mathbb{R}^{n \times 1} &= \begin{bmatrix}
                b_1 \\ b_2 \\ \vdots
            \end{bmatrix} \\
            \mathbf{C} \in \mathbb{R}^{1 \times n} &= \begin{bmatrix}
                c_1 & c_2 & \cdots
            \end{bmatrix} \\
            \dot{\mathbf{x}} &= \mathbf{A} \mathbf{x} + \mathbf{B} u \\
            y &= \mathbf{C} \mathbf{x} \\
        \end{align*}

        Bestimmung von \(\mathbf{K}\): \(\left\lvert s\mathbf{I} - \mathbf{A} + \mathbf{B} \mathbf{K} \right\rvert = 0\)

        \subsection*{Zustandsraumdarstellung in Übertragungsfunktion:}

        \subsubsection*{Rosenbrockmatrix:}
        \begin{align*}
            G(s) &= \frac{\left|\mathbf{P}(s)\right|}{\left|s\mathbf{I} - \mathbf{A} \right|} \\
            \mathbf{P}(s) &= \begin{bmatrix}
                s\mathbf{I} - \mathbf{A} & \mathbf{B} \\
                -\mathbf{C} & D
            \end{bmatrix}
        \end{align*}

        \subsubsection*{Matrixinversion:}

        \(G(s) = \det(\mathbf{C}{(s\mathbf{I} - \mathbf{A})}^{-1}\mathbf{B})+D\)

        \section*{Regelungsnormalform}
        \begin{align*}
            G(s) &= \frac{b_1 s^{n-1} + \cdots + b_{n-2} s^2 + b_{n-1} s + b_n}{s^n + a_1 s^{n-1} + \cdots + a_{n-2} s^2 + a_{n-1} s + a_n} \\
            \mathbf{A} &= \begin{bmatrix}
                0 & 1 & 0 & \cdots & 0 \\
                0 & 0 & 1 & \cdots & 0 \\
                \vdots & \vdots & \vdots & \ddots & \vdots \\
                0 & 0 & 0 & \cdots & 1 \\
                -a_n & -a_{n-1} & -a_{n-2} & \cdots & -a_1
            \end{bmatrix} \\
            \mathbf{B} &= \begin{bmatrix}
                0 \\ 0 \\ \vdots \\ 1
            \end{bmatrix} \\
            \mathbf{C} &= \begin{bmatrix}
                b_n & b_{n-1} & \cdots & b_1
            \end{bmatrix} \\
            \dot{\mathbf{x}} &= \mathbf{A} \mathbf{x} + \mathbf{B} u \\
            y &= \mathbf{C} \mathbf{x} \\
        \end{align*}

        \section*{Beobachtungsnormalform}

        Wie Regelungsnormalform, nur \(\mathbf{A} = \mathbf{A}_{RNF}^T; \mathbf{B} = \mathbf{C}_{RNF}^T; \mathbf{C} = \mathbf{B}_{RNF}^T\)

        \section*{Jordannormalform}

        Polynom muss in Partialbrüchen vorliegen. Betrachtung jedes Terms einzeln.

        \begin{itemize}
            \item \(\pm \frac{b}{s+p}\): \(A_{c_i} = \begin{bmatrix}-p\end{bmatrix}\), \(B_{c_i} = \begin{bmatrix}1\end{bmatrix}\), \(C_{c_i} = \begin{bmatrix}\pm b\end{bmatrix}\)
            \item \(\frac{b}{{(s-p)}^m}\): \(m\) Spalten und Zeilen
            
            \(A_{c_i} = \begin{bmatrix}
                p & 1 & 0 & \cdots & 0 \\
                0 & p & 1 & \cdots & 0 \\
                \vdots & \vdots & \ddots & \ddots & \vdots \\
                0 & 0 & 0 & \ddots & 1 \\
                0 & 0 & 0 & \cdots & p \\
            \end{bmatrix}\), \(B_{c_i} = \begin{bmatrix}0 \\ 0 \\ 0 \\ \vdots \\ 1\end{bmatrix}\),
            
            \(C_{c_i} = \begin{bmatrix}b & 0 & 0 & \cdots & 0\end{bmatrix}\)
            \item \(\frac{b_1 s + b_2}{s^2 + a_1 s + a_2}\):  \(A_{c_i} = \begin{bmatrix}
                0 & 1 \\
                -a_2 & -a_1 \\
            \end{bmatrix}\), \(B_{c_i} = \begin{bmatrix}0 \\ 1\end{bmatrix}\), \(C_{c_i} = \begin{bmatrix}b_2 & b_1\end{bmatrix}\)
        \end{itemize}

        Dann alles zusammenfügen:
        \begin{align*}
            A_d &= \begin{bmatrix}
                A_{c_1} & 0 & \cdots & 0 \\
                0 & A_{c_2} & \cdots & 0 \\
                \vdots & \vdots & \ddots & \vdots \\
                0 & 0 & \cdots & A_{c_n} \\
            \end{bmatrix} \\
            B_d &= \begin{bmatrix}
                B_{c_1} \\ B_{c_2} \\ \vdots \\ B_{c_n}
            \end{bmatrix} \\
            C_d &= \begin{bmatrix}
                C_{c_1} & C_{c_2} & \cdots & C_{c_n}
            \end{bmatrix}
        \end{align*}

        \section*{Reglerauslegung}

        Gleichung \(G(s) = \frac{b(s)}{a(s)}\) gleichsetzen: \(a(s) = s^2+2\zeta\omega_0 s + {\omega_0}^2\)

        Anregelzeit \(t_r \approx \frac{1.8}{\omega_0}\)

        \(\omega_d = \omega_0 \sqrt{1-\zeta^2}\)

        Anstiegzeit \(t_p = \frac{\pi}{\omega_d}\)

        Überschwingweite \(M_p = e^{-\frac{\pi\zeta}{\sqrt{1-\zeta^2}}}\)

        Einschwingzeit \(t_{s,1\%} = \frac{-\ln(0.01)}{\zeta\omega_0}\)

        Sprungantwort
        \[y(t) = 1-e^{-\zeta\omega_0 t}\left(\cos(\omega_d t) + \frac{\zeta\omega_0}{\omega_d}\sin(\omega_d t)\right)\]

        \section*{Folgeregelung}
        \begin{align*}
            \begin{bmatrix} A & B \\ C & D\end{bmatrix} \begin{bmatrix}N_x \\ N_u\end{bmatrix} &= \begin{bmatrix} b_1 \\ b_2 \end{bmatrix} \\
                \mathbf{\bar{N}} &= N_u + K \mathbf{N_x} \\
            u &= -\mathbf{K}\mathbf{x} + \mathbf{\bar{N}}r
        \end{align*}

        \section*{Bode-Diagramm}

        Konstante: \(G(s)=K\frac{\dots\dots}{\dots\dots} \Rrightarrow \) Betrag: \(20 \log_{10}(K)\); Phase: \(\begin{cases}
            0\degree & K > 0 \\
            -180\degree & K < 0
        \end{cases}\)

        Pol im Ursprung: \(G(s) = \frac{\dots\dots}{s(\dots)}\) Betrag: \(-20 \frac{dB}{Dekade}\) um \(\omega = 1\); Phase: konstant -90\degree

        Nullstelle im Ursprung: \(G(s) = \frac{s(\dots)}{\dots\dots}\) Betrag: \(+20 \frac{dB}{Dekade}\) um \(\omega = 1\); Phase: konstant +90\degree

        reeller Pol: \(G(s) = \frac{\dots\dots}{(s+\omega_l)(\dots)}\) Betrag: \(-20 \frac{dB}{Dekade}\) ab \(\omega = \omega_l\); Phase: Pol links der imaginären Achse: Fall A, rechts der imaginären Achse Fall B

        reelle Nullstelle: \(G(s) = \frac{(s+\omega_l)(\dots)}{\dots\dots}\) Betrag: \(+20 \frac{dB}{Dekade}\) ab \(\omega = \omega_l\); Phase: Nullstelle links der imaginären Achse: Fall B, rechts der imaginären Achse Fall A

        Fall A: von 0\degree\ auf -90\degree\ linear von \(0.1\omega_l\) bis \(10\omega_l\)

        Fall B: von 0\degree\ auf +90\degree\ linear von \(0.1\omega_l\) bis \(10\omega_l\)

        Bei doppelten Null- und Polstellen Verdopplung von Betrag und Phase (40 statt 20 und 180 statt 90)

        Summe aller Terme ergibt Gesamtlösung

        \section*{Steuerbarkeit}

        Regelungsnormalform, wenn die Steuerbarkeitsmatrix \(\det(\begin{bmatrix}
        \mathbf{B} & \mathbf{A} \mathbf{B} & \mathbf{A}^2 \mathbf{B} & \cdots & \mathbf{A}^{n-1} \mathbf{B}\end{bmatrix}) \ne 0 \to\) System steuerbar (n ist Anzahl Zeilen von \(\mathbf{A}\))

        \section*{Beobachtbarkeit}

        Beobachtungsnormalform, wenn die Beobachtbarkeitsmatrix \(\det\left(\begin{bmatrix}
        \mathbf{C} \\ \mathbf{C}\mathbf{A} \\ \mathbf{C}\mathbf{A}^2 \\ \vdots \\ \mathbf{C}\mathbf{A}^{n-1} \end{bmatrix}\right) \ne 0 \to\) System beobachtbar (n ist Anzahl Zeilen von \(\mathbf{A}\))

        \section*{Beobachter}

        Charakteristische Gleichung: \(\det(s\mathbf{I} - \mathbf{A} + \mathbf{L} \mathbf{C} )\)

    \end{multicols}
 
\end{document}          